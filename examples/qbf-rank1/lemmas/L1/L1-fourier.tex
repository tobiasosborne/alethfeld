% Alethfeld Generated LaTeX
% Lemma: L1-fourier (Fourier Coefficient Formula)
% Graph Version: verified-2024
% Status: VERIFIED | Taint: CLEAN
% Generated by Alethfeld Orchestrator v4

\documentclass[11pt]{article}
\usepackage{amsmath,amssymb,amsthm}
\usepackage{braket}
\usepackage{geometry}
\usepackage{enumitem}
\usepackage{xcolor}

\geometry{margin=1in}

\theoremstyle{plain}
\newtheorem{lemma}{Lemma}
\newtheorem{claim}{Claim}
\theoremstyle{definition}
\newtheorem{definition}{Definition}
\newtheorem{assumption}{Assumption}

\newcommand{\Tr}{\operatorname{Tr}}

\title{Lemma L1: Fourier Coefficient Formula\\[0.5em]
\large Alethfeld Verified Proof}
\author{Alethfeld Proof Orchestrator v4}
\date{Verification Status: \textcolor{green!60!black}{\textbf{VERIFIED}} $\mid$ Taint: \textcolor{green!60!black}{\textbf{CLEAN}}}

\begin{document}
\maketitle

\section*{Dependencies}

\begin{assumption}[A1: Product State QBF]\label{ass:A1}
Let $U = I - 2\ket{\psi}\bra{\psi}$ be a rank-1 quantum Boolean function where $\ket{\psi} = \bigotimes_{k=1}^n \ket{\phi_k}$ is a product state.
\end{assumption}

\begin{definition}[D1: Bloch Vector]\label{def:D1}
For each qubit $k$, the Bloch vector $\vec{r}_k = (x_k, y_k, z_k)$ satisfies $|\vec{r}_k|^2 = 1$. Define extended Bloch components:
\[
r_k^{(0)} = 1, \quad r_k^{(1)} = x_k, \quad r_k^{(2)} = y_k, \quad r_k^{(3)} = z_k.
\]
\end{definition}

\section*{Statement}

\begin{lemma}[Fourier Coefficient Formula]\label{lem:L1}
Under Assumption~\ref{ass:A1} and Definition~\ref{def:D1}, for any multi-index $\alpha \in \{0,1,2,3\}^n$:
\[
\hat{U}(\alpha) = \delta_{\alpha,0} - 2^{1-n} \prod_{k=1}^n r_k^{(\alpha_k)}
\]
where $\delta_{\alpha,0} = 1$ if $\alpha = (0,\ldots,0)$ and $0$ otherwise.
\end{lemma}

\section*{Proof}

\subsection*{Step 1: Definition Expansion}

\begin{claim}[L1-step1]
$\hat{U}(\alpha) = 2^{-n}\Tr(\sigma^\alpha U) = 2^{-n}\Tr(\sigma^\alpha) - 2^{1-n}\Tr(\sigma^\alpha \ket{\psi}\bra{\psi})$
\end{claim}

\begin{proof}
By definition of the Pauli-Fourier coefficient, $\hat{U}(\alpha) = 2^{-n}\Tr(\sigma^\alpha U)$. Substituting $U = I - 2\ket{\psi}\bra{\psi}$ from Assumption~\ref{ass:A1}:
\[
\hat{U}(\alpha) = 2^{-n}\Tr(\sigma^\alpha(I - 2\ket{\psi}\bra{\psi})) = 2^{-n}\Tr(\sigma^\alpha) - 2^{1-n}\Tr(\sigma^\alpha\ket{\psi}\bra{\psi}). \qedhere
\]
\end{proof}

\subsection*{Step 2: Trace of Pauli String}

\begin{claim}[L1-step2]
$\Tr(\sigma^\alpha) = 2^n \delta_{\alpha,0}$
\end{claim}

\begin{proof}
We proceed in substeps.

\textbf{(2a)} By definition, $\sigma^\alpha = \sigma^{\alpha_1} \otimes \sigma^{\alpha_2} \otimes \cdots \otimes \sigma^{\alpha_n}$.

\textbf{(2b)} For tensor products, $\Tr(A \otimes B) = \Tr(A) \cdot \Tr(B)$.
\begin{itemize}[leftmargin=2em]
\item[(2b.1)] By definition, $(A \otimes B)_{(i,j),(k,l)} = A_{ik}B_{jl}$.
\item[(2b.2)] Thus $\Tr(A \otimes B) = \sum_{i,j}(A \otimes B)_{(i,j),(i,j)} = \sum_{i,j}A_{ii}B_{jj} = \Tr(A)\Tr(B)$.
\end{itemize}

\textbf{(2c)} Applying (2a) and (2b) inductively: $\Tr(\sigma^\alpha) = \prod_{k=1}^n \Tr(\sigma^{\alpha_k})$.

\textbf{(2d)} For single-qubit Pauli matrices:
\begin{itemize}[leftmargin=2em]
\item[(2d.1)] $\sigma^0 = I_2 = \begin{pmatrix}1 & 0\\ 0 & 1\end{pmatrix}$, so $\Tr(\sigma^0) = 2$.
\item[(2d.2)] $\sigma^1 = \sigma_x = \begin{pmatrix}0 & 1\\ 1 & 0\end{pmatrix}$, so $\Tr(\sigma^1) = 0$.
\item[(2d.3)] $\sigma^2 = \sigma_y = \begin{pmatrix}0 & -i\\ i & 0\end{pmatrix}$, so $\Tr(\sigma^2) = 0$.
\item[(2d.4)] $\sigma^3 = \sigma_z = \begin{pmatrix}1 & 0\\ 0 & -1\end{pmatrix}$, so $\Tr(\sigma^3) = 0$.
\end{itemize}

\textbf{(2e)} Case analysis on the product:
\begin{itemize}[leftmargin=2em]
\item[(2e.1)] If $\exists k: \alpha_k \neq 0$, then $\Tr(\sigma^{\alpha_k}) = 0$, so the product vanishes.
\item[(2e.2)] If $\forall k: \alpha_k = 0$, then each factor equals $2$, giving $2^n$.
\end{itemize}

Therefore $\Tr(\sigma^\alpha) = 2^n \delta_{\alpha,0}$.
\end{proof}

\subsection*{Step 3: Trace Cyclic Property}

\begin{claim}[L1-step3]
$\Tr(\sigma^\alpha \ket{\psi}\bra{\psi}) = \braket{\psi|\sigma^\alpha|\psi}$
\end{claim}

\begin{proof}
\textbf{(3a)} The operator $\ket{\psi}\bra{\psi}$ is a rank-1 projector.

\textbf{(3c)} For any operator $A$ and normalized state $\ket{\psi}$:
\begin{itemize}[leftmargin=2em]
\item[(3c.1)] Let $\{\ket{e_j}\}$ be an orthonormal basis. Then
\[
\Tr(A\ket{\psi}\bra{\psi}) = \sum_j \braket{e_j|A|\psi}\braket{\psi|e_j}.
\]
\item[(3c.2)] Rearranging scalars: $= \sum_j \braket{\psi|e_j}\braket{e_j|A|\psi}$.
\item[(3c.3)] By completeness ($\sum_j \ket{e_j}\bra{e_j} = I$):
\[
= \bra{\psi}\left(\sum_j \ket{e_j}\bra{e_j}\right)A\ket{\psi} = \braket{\psi|A|\psi}.
\]
\end{itemize}

Setting $A = \sigma^\alpha$ yields the result.
\end{proof}

\subsection*{Step 4: Product State Factorization}

\begin{claim}[L1-step4]
$\braket{\psi|\sigma^\alpha|\psi} = \prod_{k=1}^n \braket{\phi_k|\sigma^{\alpha_k}|\phi_k} = \prod_{k=1}^n r_k^{(\alpha_k)}$
\end{claim}

\begin{proof}
\textbf{(4a)} From Assumption~\ref{ass:A1}: $\ket{\psi} = \ket{\phi_1} \otimes \cdots \otimes \ket{\phi_n}$ and $\sigma^\alpha = \sigma^{\alpha_1} \otimes \cdots \otimes \sigma^{\alpha_n}$.

\textbf{(4c)} Tensor product operator action: $(A \otimes B)(\ket{a} \otimes \ket{b}) = (A\ket{a}) \otimes (B\ket{b})$.

\textbf{(4d)} Tensor product inner product: $(\bra{a} \otimes \bra{b})(\ket{c} \otimes \ket{d}) = \braket{a|c}\braket{b|d}$.

\textbf{(4e)} Combining (4c) and (4d) inductively:
\[
\braket{\psi|\sigma^\alpha|\psi} = \prod_{k=1}^n \braket{\phi_k|\sigma^{\alpha_k}|\phi_k}.
\]

\textbf{(4f)} For a single qubit $\ket{\phi}$ with Bloch vector $(x,y,z)$, parametrized as $\ket{\phi} = \cos(\theta/2)\ket{0} + e^{i\varphi}\sin(\theta/2)\ket{1}$ where $x = \sin\theta\cos\varphi$, $y = \sin\theta\sin\varphi$, $z = \cos\theta$:

\begin{itemize}[leftmargin=2em]
\item[(4f.2)] $\braket{\phi|I|\phi} = 1 = r^{(0)}$. \checkmark
\item[(4f.3)] $\braket{\phi|\sigma_x|\phi}$: Since $\sigma_x\ket{\phi} = \cos(\theta/2)\ket{1} + e^{i\varphi}\sin(\theta/2)\ket{0}$,
\begin{align*}
\braket{\phi|\sigma_x|\phi} &= e^{-i\varphi}\sin(\theta/2)\cos(\theta/2) + e^{i\varphi}\cos(\theta/2)\sin(\theta/2)\\
&= 2\cos(\theta/2)\sin(\theta/2)\cos\varphi = \sin\theta\cos\varphi = x = r^{(1)}. \checkmark
\end{align*}
\item[(4f.4)] $\braket{\phi|\sigma_y|\phi} = \sin\theta\sin\varphi = y = r^{(2)}$. (Analogous calculation.)
\item[(4f.5)] $\braket{\phi|\sigma_z|\phi}$: Since $\sigma_z\ket{0} = \ket{0}$ and $\sigma_z\ket{1} = -\ket{1}$,
\[
\braket{\phi|\sigma_z|\phi} = \cos^2(\theta/2) - \sin^2(\theta/2) = \cos\theta = z = r^{(3)}. \checkmark
\]
\end{itemize}

Therefore $\braket{\phi_k|\sigma^{\alpha_k}|\phi_k} = r_k^{(\alpha_k)}$ for each $k$.
\end{proof}

\subsection*{Conclusion}

\begin{proof}[Proof of Lemma~\ref{lem:L1}]
Combining Steps 1--4:
\begin{align*}
\hat{U}(\alpha) &= 2^{-n}\Tr(\sigma^\alpha) - 2^{1-n}\Tr(\sigma^\alpha\ket{\psi}\bra{\psi}) && \text{(Step 1)}\\
&= 2^{-n} \cdot 2^n\delta_{\alpha,0} - 2^{1-n}\braket{\psi|\sigma^\alpha|\psi} && \text{(Steps 2, 3)}\\
&= \delta_{\alpha,0} - 2^{1-n}\prod_{k=1}^n r_k^{(\alpha_k)} && \text{(Step 4)}
\end{align*}
as claimed. \qed
\end{proof}

\vfill
\hrule
\vspace{0.5em}
\noindent\textbf{Alethfeld Verification Report}\\
\begin{tabular}{@{}ll@{}}
Graph ID: & L1-fourier-verify-2024\\
Total Nodes: & 46 (6 original + 40 expanded)\\
Max Depth: & 5\\
Admitted Steps: & 0\\
Taint Status: & \textcolor{green!60!black}{CLEAN}\\
\end{tabular}

\end{document}
