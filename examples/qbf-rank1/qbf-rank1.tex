\documentclass[11pt,a4paper]{article}

\usepackage{amsmath,amssymb,amsthm}
\usepackage{mathtools}
\usepackage{braket}
\usepackage{hyperref}
\usepackage{cleveref}
\usepackage{enumitem}
\usepackage{booktabs}
\usepackage{geometry}
\geometry{margin=1in}

% Theorem environments (Lamport-style numbering)
\newtheorem{theorem}{Theorem}[section]
\newtheorem{lemma}[theorem]{Lemma}
\newtheorem{corollary}[theorem]{Corollary}
\newtheorem{proposition}[theorem]{Proposition}
\newtheorem{definition}[theorem]{Definition}

\theoremstyle{remark}
\newtheorem*{remark}{Remark}

% Custom commands
\newcommand{\R}{\mathbb{R}}
\newcommand{\C}{\mathbb{C}}
\newcommand{\N}{\mathbb{N}}
\newcommand{\Tr}{\operatorname{Tr}}
\newcommand{\sgn}{\operatorname{sgn}}

% Alethfeld metadata
\newcommand{\alethfeldid}[1]{\textsuperscript{\tiny\texttt{#1}}}

\title{Entropy-Influence Bound for Rank-1 Product State QBFs}
\author{Alethfeld Proof System v4}
\date{Graph ID: \texttt{qbf-rank1-entropy-influence} \\ Version: 2 (Verified)}

\begin{document}

\maketitle

\begin{abstract}
We establish an explicit upper bound on the entropy-to-influence ratio for rank-1 quantum Boolean functions (QBFs) constructed from product states. For the QBF $U = I - 2\ket{\psi}\bra{\psi}$ where $\ket{\psi} = \bigotimes_{k=1}^n \ket{\phi_k}$ is a product state, we prove that the influence $I(U) = n \cdot 2^{1-n}$ is independent of the choice of single-qubit states, while the entropy $S(U)$ is maximized when all qubits are in the ``magic'' state with Bloch vector $(1/\sqrt{3}, 1/\sqrt{3}, 1/\sqrt{3})$. The ratio $S/I$ approaches $\log_2 3 + 4 \approx 5.585$ as $n \to \infty$, establishing a lower bound on any universal constant $C$ satisfying $S(U) \leq C \cdot I(U)$.
\end{abstract}

\tableofcontents

%═══════════════════════════════════════════════════════════════
\section{Main Result}
%═══════════════════════════════════════════════════════════════

\begin{theorem}[Entropy-Influence Bound]\label{thm:main}\alethfeldid{:theorem}
For the rank-1 QBF $U = I - 2\ket{\psi}\bra{\psi}$ where $\ket{\psi} = \bigotimes_{k=1}^n \ket{\phi_k}$ is a product state:
\begin{equation}
\frac{S(U)}{I(U)} \leq \log_2 3 + \frac{2^{n-1}}{n}\left[-p_0 \log_2 p_0 + (2n-2)(1-p_0)\right]
\end{equation}
where $p_0 = (1 - 2^{1-n})^2$. The maximum is achieved when all qubits are in the magic state with Bloch vector $\left(\frac{1}{\sqrt{3}}, \frac{1}{\sqrt{3}}, \frac{1}{\sqrt{3}}\right)$.
\end{theorem}

The proof proceeds through six parts: establishing Fourier coefficients, computing influence, deriving the entropy formula, identifying the maximum, analyzing asymptotics, and drawing implications for the entropy-influence conjecture.

%═══════════════════════════════════════════════════════════════
\section{Preliminaries}
%═══════════════════════════════════════════════════════════════

\subsection{Setup}\alethfeldid{:0-assume0}

Let $U = I - 2\ket{\psi}\bra{\psi}$ be a rank-1 QBF where $\ket{\psi} = \bigotimes_{k=1}^n \ket{\phi_k}$ is a product state with each $\ket{\phi_k} \in \C^2$.

\begin{definition}[Bloch Vector]\label{def:bloch}\alethfeldid{:1-def1}
Each single-qubit state $\ket{\phi_k}$ has Bloch vector $\vec{r}_k = (x_k, y_k, z_k)$ with
\begin{equation}
|\vec{r}_k|^2 = x_k^2 + y_k^2 + z_k^2 = 1.
\end{equation}
\end{definition}

\begin{definition}[Extended Bloch Coefficients]\label{def:qk}\alethfeldid{:1-def2}
Define $q_k^{(0)} = 1$ and
\begin{equation}
\left(q_k^{(1)}, q_k^{(2)}, q_k^{(3)}\right) = \left(x_k^2, y_k^2, z_k^2\right).
\end{equation}
\end{definition}

\begin{definition}[Bloch Entropy]\label{def:bloch-entropy}\alethfeldid{:1-def3}
The Bloch entropy of qubit $k$ is
\begin{equation}
f_k = H(x_k^2, y_k^2, z_k^2) = -\sum_{\ell=1}^3 q_k^{(\ell)} \log_2 q_k^{(\ell)}.
\end{equation}
\end{definition}

\begin{remark}
The Bloch entropy $f_k$ measures the ``spread'' of the Bloch vector across coordinate axes. It is \emph{not} the von Neumann entropy of the qubit state (which is zero for pure states).
\end{remark}

%═══════════════════════════════════════════════════════════════
\section{Fourier Coefficients}
%═══════════════════════════════════════════════════════════════

\begin{lemma}[Fourier Coefficient Formula]\label{lem:fourier}\alethfeldid{:1-lem1}
For $U = I - 2\ket{\psi}\bra{\psi}$:
\begin{equation}
\hat{U}(\alpha) = \delta_{\alpha,0} - 2^{1-n} \prod_{k=1}^n r_k^{(\alpha_k)}
\end{equation}
where $r_k^{(0)} = 1$, $r_k^{(1)} = x_k$, $r_k^{(2)} = y_k$, $r_k^{(3)} = z_k$.
\end{lemma}

\begin{proof}
\begin{enumerate}[label=\textbf{\arabic*.}, ref=\arabic*, leftmargin=*]
\item\alethfeldid{:2-lem1-1} By definition of the Pauli-Fourier expansion:
\begin{equation}
\hat{U}(\alpha) = 2^{-n}\Tr(\sigma^\alpha U) = 2^{-n}\Tr(\sigma^\alpha) - 2^{1-n}\Tr(\sigma^\alpha \ket{\psi}\bra{\psi}).
\end{equation}

\item\alethfeldid{:2-lem1-2} The trace of Pauli strings satisfies:
\begin{equation}
\Tr(\sigma^\alpha) = 2^n \delta_{\alpha,0}
\end{equation}
since $\Tr(\sigma_i) = 0$ for $i \in \{1,2,3\}$ and $\Tr(I) = 2$.

\item\alethfeldid{:2-lem1-3} By the cyclic property of trace:
\begin{equation}
\Tr(\sigma^\alpha \ket{\psi}\bra{\psi}) = \bra{\psi}\sigma^\alpha\ket{\psi}.
\end{equation}

\item\alethfeldid{:2-lem1-4} For a product state, this expectation value factorizes:
\begin{equation}
\bra{\psi}\sigma^\alpha\ket{\psi} = \prod_k \bra{\phi_k}\sigma^{\alpha_k}\ket{\phi_k} = \prod_k r_k^{(\alpha_k)}.
\end{equation}
\end{enumerate}
Combining these steps yields the result.\alethfeldid{:2-lem1-qed}
\end{proof}

\begin{lemma}[Probability Distribution]\label{lem:prob}\alethfeldid{:1-lem2}
The Fourier weight distribution is:
\begin{equation}
p_\alpha = |\hat{U}(\alpha)|^2 = \begin{cases}
(1 - 2^{1-n})^2 & \alpha = 0 \\[4pt]
2^{2-2n} \displaystyle\prod_{k=1}^n q_k^{(\alpha_k)} & \alpha \neq 0
\end{cases}
\end{equation}
\end{lemma}

\begin{proof}
For $\alpha = 0$: $|\hat{U}(0)|^2 = |1 - 2^{1-n}|^2 = (1 - 2^{1-n})^2$.

For $\alpha \neq 0$: $|\hat{U}(\alpha)|^2 = |{-}2^{1-n} \prod_k r_k^{(\alpha_k)}|^2 = 2^{2-2n} \prod_k |r_k^{(\alpha_k)}|^2 = 2^{2-2n} \prod_k q_k^{(\alpha_k)}$.
\end{proof}

%═══════════════════════════════════════════════════════════════
\section{Influence Calculation}
%═══════════════════════════════════════════════════════════════

\begin{theorem}[Influence Independence]\label{thm:influence}\alethfeldid{:1-thm3}
For any rank-1 product state QBF:
\begin{equation}
I(U) = n \cdot 2^{1-n}.
\end{equation}
This is \textbf{independent of the choice of Bloch vectors}.
\end{theorem}

\begin{proof}
\begin{enumerate}[label=\textbf{\arabic*.}, ref=\arabic*, leftmargin=*]
\item\alethfeldid{:2-thm3-1} The influence of qubit $j$ is defined as:
\begin{equation}
I_j = \sum_{\alpha: \alpha_j \neq 0} p_\alpha.
\end{equation}

\item\alethfeldid{:2-thm3-2} For $\alpha \neq 0$ with $\alpha_j = \ell \neq 0$, we sum over all choices of $(\alpha_k)_{k \neq j} \in \{0,1,2,3\}^{n-1}$:
\begin{align}
\sum_{\alpha: \alpha_j = \ell} p_\alpha &= \sum_{(\alpha_k)_{k \neq j}} 2^{2-2n} \cdot q_j^{(\ell)} \cdot \prod_{k \neq j} q_k^{(\alpha_k)} \\
&= 2^{2-2n} \cdot q_j^{(\ell)} \cdot \prod_{k \neq j} \left(\sum_{m=0}^3 q_k^{(m)}\right).
\end{align}

\item\alethfeldid{:2-thm3-3} Since $\sum_{m=0}^3 q_k^{(m)} = 1 + x_k^2 + y_k^2 + z_k^2 = 1 + 1 = 2$:
\begin{equation}
\sum_{\alpha: \alpha_j = \ell} p_\alpha = 2^{2-2n} \cdot q_j^{(\ell)} \cdot 2^{n-1}.
\end{equation}

\item\alethfeldid{:2-thm3-4} Summing over $\ell \in \{1,2,3\}$:
\begin{align}
I_j &= \sum_{\ell=1}^3 2^{2-2n} \cdot 2^{n-1} \cdot q_j^{(\ell)} \\
&= 2^{1-n} \cdot \sum_{\ell=1}^3 q_j^{(\ell)} \\
&= 2^{1-n} \cdot (x_j^2 + y_j^2 + z_j^2) \\
&= 2^{1-n} \cdot 1 = 2^{1-n}.
\end{align}
\end{enumerate}
Therefore, the total influence is:\alethfeldid{:2-thm3-qed}
\begin{equation}
I = \sum_{j=1}^n I_j = n \cdot 2^{1-n}. \qedhere
\end{equation}
\end{proof}

%═══════════════════════════════════════════════════════════════
\section{Entropy Calculation}
%═══════════════════════════════════════════════════════════════

\begin{lemma}[Entropy Decomposition]\label{lem:entropy-decomp}\alethfeldid{:1-lem4}
\begin{equation}
S = -p_0 \log_2 p_0 - \sum_{\alpha \neq 0} p_\alpha \log_2 p_\alpha.
\end{equation}
\end{lemma}

\begin{theorem}[General Entropy Formula]\label{thm:entropy}\alethfeldid{:1-thm5}
\begin{equation}
S = -p_0 \log_2 p_0 + (2n-2)(1-p_0) + 2^{1-n} \sum_{k=1}^n f_k
\end{equation}
where $f_k = H(x_k^2, y_k^2, z_k^2)$ is the Bloch entropy of qubit $k$.
\end{theorem}

\begin{proof}
\begin{enumerate}[label=\textbf{\arabic*.}, ref=\arabic*, leftmargin=*]
\item\alethfeldid{:2-thm5-1} For $\alpha \neq 0$:
\begin{align}
-p_\alpha \log_2 p_\alpha &= -p_\alpha \log_2\left(2^{2-2n} \prod_k q_k^{(\alpha_k)}\right) \\
&= -p_\alpha \left[(2-2n)\log_2 2 + \sum_k \log_2 q_k^{(\alpha_k)}\right] \\
&= p_\alpha(2n-2) - p_\alpha \sum_k \log_2 q_k^{(\alpha_k)}.
\end{align}

\item\alethfeldid{:2-thm5-2} Summing over all $\alpha \neq 0$:
\begin{equation}
\sum_{\alpha \neq 0} p_\alpha(2n-2) = (2n-2)(1 - p_0).
\end{equation}

\item\alethfeldid{:2-thm5-3} For fixed qubit $j$, the sum $-\sum_{\alpha \neq 0} p_\alpha \log_2 q_j^{(\alpha_j)}$ splits into two cases. When $\alpha_j = 0$, we have $\log_2 q_j^{(0)} = \log_2 1 = 0$, so only $\alpha_j \neq 0$ contributes.

\item\alethfeldid{:2-thm5-4} For the nonzero part, using the result from \Cref{thm:influence}:
\begin{equation}
\sum_{\alpha: \alpha_j = \ell} p_\alpha = 2^{1-n} q_j^{(\ell)}.
\end{equation}

\item\alethfeldid{:2-thm5-5} Therefore:
\begin{align}
-\sum_{\alpha: \alpha_j \neq 0} p_\alpha \log_2 q_j^{(\alpha_j)} &= -\sum_{\ell=1}^3 \log_2 q_j^{(\ell)} \cdot 2^{1-n} q_j^{(\ell)} \\
&= -2^{1-n} \sum_{\ell=1}^3 q_j^{(\ell)} \log_2 q_j^{(\ell)} \\
&= 2^{1-n} f_j.
\end{align}

\item\alethfeldid{:2-thm5-6} Summing over all qubits $j$:
\begin{equation}
-\sum_{\alpha \neq 0} p_\alpha \sum_k \log_2 q_k^{(\alpha_k)} = 2^{1-n} \sum_{k=1}^n f_k.
\end{equation}
\end{enumerate}
Combining all terms:\alethfeldid{:2-thm5-qed}
\begin{equation}
S = -p_0 \log_2 p_0 + (2n-2)(1-p_0) + 2^{1-n} \sum_{k=1}^n f_k. \qedhere
\end{equation}
\end{proof}

%═══════════════════════════════════════════════════════════════
\section{Maximum at Magic State}
%═══════════════════════════════════════════════════════════════

\begin{theorem}[Maximum Ratio]\label{thm:maximum}\alethfeldid{:1-thm6}
The ratio $S/I$ is maximized when all qubits are in the magic state
\begin{equation}
(x_k^2, y_k^2, z_k^2) = \left(\frac{1}{3}, \frac{1}{3}, \frac{1}{3}\right).
\end{equation}
\end{theorem}

\begin{proof}
\begin{enumerate}[label=\textbf{\arabic*.}, ref=\arabic*, leftmargin=*]
\item\alethfeldid{:2-thm6-1} Since $I = n \cdot 2^{1-n}$ is constant (independent of Bloch vectors by \Cref{thm:influence}), maximizing $S/I$ is equivalent to maximizing $S$.

\item\alethfeldid{:2-thm6-2} From \Cref{thm:entropy}:
\begin{equation}
S = -p_0 \log_2 p_0 + (2n-2)(1-p_0) + 2^{1-n} \sum_{k=1}^n f_k.
\end{equation}
The first two terms depend only on $n$, not on the Bloch vectors.

\item\alethfeldid{:2-thm6-3} Each $f_k = H(x_k^2, y_k^2, z_k^2)$ is the Shannon entropy of a probability distribution on 3 outcomes (since $x_k^2 + y_k^2 + z_k^2 = 1$).

\item\alethfeldid{:2-thm6-4} By the maximum entropy principle \cite{Shannon1948}, for a probability distribution on $k$ outcomes:
\begin{equation}
H(p_1, \ldots, p_k) \leq \log_2 k
\end{equation}
with equality if and only if $p_i = 1/k$ for all $i$. Applied to $k=3$:
\begin{equation}
f_k \leq \log_2 3
\end{equation}
with equality if and only if $(x_k^2, y_k^2, z_k^2) = (1/3, 1/3, 1/3)$.
\end{enumerate}
Therefore $S$ is maximized when all $f_k = \log_2 3$.\alethfeldid{:2-thm6-qed}
\end{proof}

\begin{corollary}[Explicit Maximum]\label{cor:explicit}\alethfeldid{:1-cor7}
For the symmetric magic product state:
\begin{equation}
\frac{S}{I} = \log_2 3 + \frac{2^{n-1}}{n}\left[-p_0 \log_2 p_0 + (2n-2)(1-p_0)\right]
\end{equation}
where $p_0 = (1 - 2^{1-n})^2$.
\end{corollary}

\begin{proof}
At the magic state, $f_k = \log_2 3$ for all $k$. Substituting into the entropy formula:
\begin{align}
S_{\max} &= -p_0 \log_2 p_0 + (2n-2)(1-p_0) + 2^{1-n} \cdot n \cdot \log_2 3.
\end{align}
Dividing by $I = n \cdot 2^{1-n}$:
\begin{align}
\frac{S_{\max}}{I} &= \frac{-p_0 \log_2 p_0 + (2n-2)(1-p_0)}{n \cdot 2^{1-n}} + \log_2 3 \\
&= \log_2 3 + \frac{2^{n-1}}{n}\left[-p_0 \log_2 p_0 + (2n-2)(1-p_0)\right]. \qedhere
\end{align}
\end{proof}

%═══════════════════════════════════════════════════════════════
\section{Asymptotic Analysis}
%═══════════════════════════════════════════════════════════════

\begin{theorem}[Limiting Behavior]\label{thm:limit}\alethfeldid{:1-thm8}
\begin{equation}
\lim_{n \to \infty} \frac{S_{\max}}{I} = \log_2 3 + 4 \approx 5.585.
\end{equation}
\end{theorem}

\begin{proof}
Let $\varepsilon = 2^{1-n}$.\alethfeldid{:2-thm8-0} Then $p_0 = (1-\varepsilon)^2$ and $1 - p_0 = 2\varepsilon - \varepsilon^2 \approx 2\varepsilon$ for small $\varepsilon$.

\begin{enumerate}[label=\textbf{\arabic*.}, ref=\arabic*, leftmargin=*]
\item\alethfeldid{:2-thm8-1} For the entropy term, using $\log_2(1-x) \approx -x/\ln 2$ for small $x$:
\begin{equation}
-p_0 \log_2 p_0 \approx -(1-2\varepsilon)\left(\frac{-2\varepsilon}{\ln 2}\right) \approx \frac{2\varepsilon}{\ln 2}.
\end{equation}

\item\alethfeldid{:2-thm8-2} For the influence term:
\begin{equation}
(2n-2)(1-p_0) \approx (2n-2) \cdot 2\varepsilon = 4(n-1)\varepsilon.
\end{equation}

\item\alethfeldid{:2-thm8-3} The correction term becomes:
\begin{align}
g(n) &= \frac{2^{n-1}}{n}\left[-p_0 \log_2 p_0 + (2n-2)(1-p_0)\right] \\
&\approx \frac{2^{n-1}}{n} \cdot \varepsilon \cdot \left[\frac{2}{\ln 2} + 4(n-1)\right] \\
&= \frac{2^{n-1}}{n} \cdot 2^{1-n} \cdot \left[\frac{2}{\ln 2} + 4(n-1)\right].
\end{align}

\item\alethfeldid{:2-thm8-4} Simplifying:
\begin{align}
g(n) &= \frac{1}{n}\left[\frac{2}{\ln 2} + 4(n-1)\right] \\
&= \frac{2}{n \ln 2} + 4 - \frac{4}{n} \\
&\to 0 + 4 - 0 = 4 \quad \text{as } n \to \infty.
\end{align}
\end{enumerate}
Therefore:\alethfeldid{:2-thm8-qed}
\begin{equation}
\frac{S_{\max}}{I} \to \log_2 3 + 4 \approx 1.585 + 4 = 5.585. \qedhere
\end{equation}
\end{proof}

\begin{theorem}[Finite $n$ Values]\label{thm:finite}\alethfeldid{:1-thm9}
\begin{center}
\begin{tabular}{ccc}
\toprule
$n$ & $S_{\max}/I$ & Numerical Value \\
\midrule
1 & $\log_2 3$ & 1.585 \\
2 & $2 + \log_2 3$ & 3.585 \\
3 & (formula) & 4.541 \\
4 & (formula) & 4.987 \\
5 & (formula) & 5.209 \\
10 & (formula) & 5.469 \\
20 & (formula) & 5.529 \\
$\infty$ & $\log_2 3 + 4$ & 5.585 \\
\bottomrule
\end{tabular}
\end{center}
\end{theorem}

%═══════════════════════════════════════════════════════════════
\section{Implications for the Conjecture}
%═══════════════════════════════════════════════════════════════

\begin{theorem}[Supremum]\label{thm:supremum}\alethfeldid{:1-sup}
\begin{equation}
\sup_{n,\, \text{product states}} \frac{S}{I} = \log_2 3 + 4 \approx 5.585.
\end{equation}
This supremum is achieved in the limit $n \to \infty$ with all qubits in the magic state.
\end{theorem}

\begin{theorem}[Conjecture Bound]\label{thm:conjecture}\alethfeldid{:1-conj}
For the entropy-influence conjecture $S(U) \leq C \cdot I(U)$ to hold for all rank-1 product state QBFs, we require:
\begin{equation}
\boxed{C \geq \log_2 3 + 4 \approx 5.585}
\end{equation}
\end{theorem}

%═══════════════════════════════════════════════════════════════
\section{Conclusion}
%═══════════════════════════════════════════════════════════════

\alethfeldid{:1-main-qed}

For rank-1 QBFs from product states, we have proven:

\begin{enumerate}
\item \textbf{Influence is constant:} $I = n \cdot 2^{1-n}$ regardless of Bloch vectors.

\item \textbf{Entropy formula:}
\begin{equation}
S = -p_0 \log_2 p_0 + (2n-2)(1-p_0) + 2^{1-n} \sum_{k=1}^n f_k.
\end{equation}

\item \textbf{Maximum at magic states:} Each qubit in the $\left(\frac{1}{3}, \frac{1}{3}, \frac{1}{3}\right)$ direction (squared Bloch components).

\item \textbf{Explicit bound:}
\begin{equation}
\frac{S}{I} = \log_2 3 + \frac{2^{n-1}}{n}\left[-p_0 \log_2 p_0 + (2n-2)(1-p_0)\right].
\end{equation}

\item \textbf{Asymptotic limit:} $S/I \to \log_2 3 + 4 \approx 5.585$ as $n \to \infty$.

\item \textbf{Required constant:} Any universal bound $S \leq C \cdot I$ requires $C \geq 5.585$.
\end{enumerate}

%═══════════════════════════════════════════════════════════════
% References
%═══════════════════════════════════════════════════════════════

\begin{thebibliography}{9}
\bibitem{Shannon1948}
C.~E.~Shannon,
\emph{A Mathematical Theory of Communication},
Bell System Technical Journal, vol.~27, pp.~379--423, 623--656, 1948.
\end{thebibliography}

%═══════════════════════════════════════════════════════════════
% Alethfeld Metadata
%═══════════════════════════════════════════════════════════════

\appendix
\section{Alethfeld Graph Metadata}

\begin{verbatim}
Graph ID:      qbf-rank1-entropy-influence
Version:       2
Proof Mode:    formal-physics
Status:        VERIFIED

Nodes:         42 (42 verified, 0 proposed, 0 admitted)
Lemmas:        0 extracted
External Refs: 1 (Shannon entropy theorem)
Taint:         ALL CLEAN
Obligations:   NONE

Verification Summary:
  - Total nodes verified: 42
  - Initially accepted:   39
  - Challenged:           3
  - Revisions applied:    3
  - Final status:         ALL VERIFIED
\end{verbatim}

\end{document}
